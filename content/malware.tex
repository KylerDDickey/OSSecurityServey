\section{Android Malware}
Android malware poses a significant threat to the user base of the platform.
Smartphones are a vital tool for many and functions as a platform for work and play.
Android, being a leader in the smartphone OS market worldwide, is used by a significant number of people \cite{AndroidMarketShare2021}.
The safety of this platform is of critical importance to ensure that the users of the OS do not incur personal losses and to ensure that the platform may not be a medium for larger scale attacks on networks.

Recalling earlier discussion about the application framework layer in the Android platform \cite{AndroidDocs2022Arch}, it is clear that a large collection of powerful tools are available to the programmer.
Furthermore, it is possible to download and install applications not listed on any app stores as long as those applications fulfill the application requirements and follow the APK format \cite{AndroidDocs2022Fundamentals}
These conditions (powerful, easy to use, well documented platform software tools and an attack surface with little to no moderation) make Android a ripe target for Trojans, adware, and otherwise potentially unwanted applications (PUAs).

In a study published in 2012 about the characterization of malware targeting the Android platform, Zhou and Jiang reported that 86.0\% of the malware sample they characterized were repackaged applications containing malicious code that was added by the attackers \cite{Zhou2012}.
Furthermore, the study highlights an outbreak of a notorious Android malware named DroidKungFu.
The outbreak of DroidKungFu included the initial version of the malware, as well as several version iterations and variants.

As DroidKungFu evolved through variants and versions, it became more sophisticated.
For example, the malware inside a repackaged application could install a functional copy of the malware payload onto the device.
This ensured that the malware would remain even if a user uninstalled the original infected program from the device.
This payload was encrypted in later versions.
Encryption is a sophisticated obfuscation technique used by attackers to hide their malware.

In fact, in the same study that publicized the existence of DroidKungFu, Zhou and Jiang found that mobile antivirus services were inadequate solutions to malware detection \cite{Zhou2012}.
The software used to detect malware on the platform seemed to struggle to detect newer versions of the same malware family, due to the latter's rapid evolution and the former's use of signature-based detection.

Signature-based malware detection is considered to be largely ineffective when defending against so called "zero-day" malware, which is a fancy term for unmitigated operational malware that exists on live system, sometimes without the user's knowledge.
Signature-based malware detection schemes rely on pre-existing knowledge of a malware's existence, since a signature needs to be generated against the malware.
This gives zero-day malware time to propagate, damage, leak data, and otherwise cause harm to the platform ecosystem \cite{You2010}.

The fact that malware like DroidKungFu circumvented the techniques used by antivirus services through versioning and branching variants is a topic of concern.
It is apparent that malware authors only need to expend minimal effort to obfuscate their malware.
In the case of DroidKungFu, the release of newer versions might have only taken a matter of months.

Faruki \etal give a good overview of the types of malware that threaten the Android platform, as well as other vulnerabilities and issues with the system (which is discussed in Section IV) \cite{Faruki2015}.
This survey also highlights a number of efforts that have been made to secure the platform against threats and study them.
The study also provides a robust and clear breakdown of a number of malware families and the timeline of their emergence.
